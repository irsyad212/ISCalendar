\documentclass[11pt, letterpaper]{article}
\usepackage{listings}
\usepackage{color}

\definecolor{dkgreen}{rgb}{0,0.6,0}
\definecolor{gray}{rgb}{0.5,0.5,0.5}
\definecolor{mauve}{rgb}{0.58,0,0.82}

\lstset{frame=tb,
    language=Python,
    aboveskip=3mm,
    belowskip=3mm,
    showstringspaces=false,
    columns=flexible,
    basicstyle={\small\ttfamily},
    numbers=none,
    numberstyle=\tiny\color{gray},
    keywordstyle=\color{blue},
    commentstyle=\color{dkgreen},
    stringstyle=\color{mauve},
    breaklines=true,
    breakatwhitespace=true,
    tabsize=4
}
\begin{document}
\title{IS Calendar}
\author{Irsyad}
\date{Feb 2025}

\maketitle

\section{Introduction}
This Calendar serves to be the easiest for Mathematical Calculations on Dates. Furthermore this also can convert to any other Calendars easily.

\section{Calculations}

The suffix 'G' is for Gregorian and 'I' for IS \newline
Results for IS Calendar is $I_y$, $I_m$ and $I_d$ \newline
Results for Gregorian Calendar is $G_y$, $G_m$ and $G_d$ \newline

\section*{Gregorian to IS}

$G_y = Gmonth()$ \newline
$G_m = Gmonth()$ \newline
$G_d = Gday()$ \newline
$j = julian(G_y, G_m, G_d) - 2451544.5$ \newline
$I_y =$ $\lfloor j \div 360 \rfloor$ \newline
$a = (j \div 360) - I_y$ \newline
$I_m =$ $\lfloor a \times 12 \rfloor$ \newline
$b = (a \times 12) - I_m$ \newline
$I_d =$ $\lfloor b \times 30 \rfloor$ \newline

\section*{IS to Gregorian}

$I_y = Iyear()$ \newline
$I_m = Imonth()$ \newline
$I_d = Iday()$ \newline
$a = I_d \div 30$ \newline
$b = I_m + a$ \newline
$c = b \div 12$ \newline
$d = I_y + c$ \newline
$e = (d \times 360) + 2451544.5$ \newline
$G_y$, $G_m$, $G_d$ $=gregorian(e)$ \newline

\section{Formatting}

If the Gregorian Date is "1 Jan 2000", in IS Calendar is "0/0/0". \newline
If the Gregorian Date is "4 Feb 2025", in IS Calendar is "25/5/16". \newline
Thats Year, Month, Day formatting. \newline

\section{Others}

For the "julian" formula is: \newline
\begin{lstlisting}
def julian(yr, mn, dy):
    dy = dy - 0.5
	a = (14 - mn) // 12
	y = yr + 4800 - a
	m = mn + 12 * a - 3
	return (dy + ((153 * m + 2) // 5) + 365 * y + y // 4 - y // 100 + y // 400 - 32045) 
\end{lstlisting}
For the "gregorian" formula is: \newline
\begin{lstlisting}
def gregorian(julian):
	a = 1

	b = 1

	j = julian + 0.5
	
	i = math.floor(j)

	f = j - i

	if(i > 2299160):
		a = math.floor((i - 1867216.25)/36524.25)
		b = i + a - (a // 4) + 1
	else :
		b = i

	c = b + 1524

	d = math.floor( (c-122.1) / 365.25)

	e = math.floor(365.25 * d)

	g = math.floor( (c - e) / 30.6001 )

	day = c - e + f - math.floor(30.6001 * g)

	month = 1

	if(g < 13.5):
		month = g - 1
	else :
		month = g - 13

	year = 1

	if(month > 2.5):
		year = d - 4716
	else :
		year = d - 4715

	return year, month, day  

\end{lstlisting}
There are more uses of this calendar in the "uses" folder. \newline
It does not care about Gregorian Skip. \newline
So "15 Oct 1582" is "-423/-3/-14" \newline
And so "4 Oct 1582" is "-423/-3/-15" \newline
The "converter.py" doesn't detect the Julian to Gregorian change.
\end{document}
